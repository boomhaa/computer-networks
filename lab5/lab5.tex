\documentclass[a4paper, 14pt]{extarticle}

% Поля
%--------------------------------------
\usepackage{geometry}
\geometry{a4paper,tmargin=2cm,bmargin=2cm,lmargin=3cm,rmargin=1cm}
%--------------------------------------


%Russian-specific packages
%--------------------------------------
\usepackage[T2A]{fontenc}
\usepackage[utf8]{inputenc} 
\usepackage[english, main=russian]{babel}
%--------------------------------------

\usepackage{textcomp}

% Красная строка
%--------------------------------------
\usepackage{indentfirst}               
%--------------------------------------             


%Graphics
%--------------------------------------
\usepackage{graphicx}
\graphicspath{ {./images/} }
\usepackage{wrapfig}
%--------------------------------------

% Полуторный интервал
%--------------------------------------
\linespread{1.3}                    
%--------------------------------------

%Выравнивание и переносы
%--------------------------------------
% Избавляемся от переполнений
\sloppy
% Запрещаем разрыв страницы после первой строки абзаца
\clubpenalty=10000
% Запрещаем разрыв страницы после последней строки абзаца
\widowpenalty=10000
%--------------------------------------

%Списки
\usepackage{enumitem}

%Подписи
\usepackage{caption} 

%Гиперссылки
\usepackage{hyperref}

\hypersetup {
	unicode=true
}

%Рисунки
%--------------------------------------
\DeclareCaptionLabelSeparator*{emdash}{~--- }
\captionsetup[figure]{labelsep=emdash,font=onehalfspacing,position=bottom}
%--------------------------------------

% \usepackage{tempora}

%Листинги
%--------------------------------------
\usepackage{listings}
\lstset{
	basicstyle=\ttfamily\footnotesize, 
	%basicstyle=\footnotesize\AnkaCoder,        % the size of the fonts that are used for the code
	breakatwhitespace=false,         % sets if automatic breaks shoulbd only happen at whitespace
	breaklines=true,                 % sets automatic line breaking
	captionpos=t,                    % sets the caption-position to bottom
	inputencoding=utf8,
	frame=single,                    % adds a frame around the code
	keepspaces=true,                 % keeps spaces in text, useful for keeping indentation of code (possibly needs columns=flexible)
	keywordstyle=\bf,       % keyword style
	numbers=left,                    % where to put the line-numbers; possible values are (none, left, right)
	numbersep=5pt,                   % how far the line-numbers are from the code
	xleftmargin=25pt,
	xrightmargin=25pt,
	showspaces=false,                % show spaces everywhere adding particular underscores; it overrides 'showstringspaces'
	showstringspaces=false,          % underline spaces within strings only
	showtabs=false,                  % show tabs within strings adding particular underscores
	stepnumber=1,                    % the step between two line-numbers. If it's 1, each line will be numbered
	tabsize=2,                       % sets default tabsize to 8 spaces
	title=\lstname                   % show the filename of files included with \lstinputlisting; also try caption instead of title
}
%--------------------------------------

%%% Математические пакеты %%%
%--------------------------------------
\usepackage{amsthm,amsfonts,amsmath,amssymb,amscd}  % Математические дополнения от AMS
\usepackage{mathtools}                              % Добавляет окружение multlined
\usepackage[perpage]{footmisc}
%--------------------------------------

%--------------------------------------
%			НАЧАЛО ДОКУМЕНТА
%--------------------------------------

\begin{document}
	
	%--------------------------------------
	%			ТИТУЛЬНЫЙ ЛИСТ
	%--------------------------------------
	\begin{titlepage}
		\thispagestyle{empty}
		\newpage
		
		
		%Шапка титульного листа
		%--------------------------------------
		\vspace*{-60pt}
		\hspace{-65pt}
		\begin{minipage}{0.3\textwidth}
			\hspace*{-20pt}\centering
			\includegraphics[width=\textwidth]{emblem}
		\end{minipage}
		\begin{minipage}{0.67\textwidth}\small \textbf{
				\vspace*{-0.7ex}
				\hspace*{-6pt}\centerline{Министерство науки и высшего образования Российской Федерации}
				\vspace*{-0.7ex}
				\centerline{Федеральное государственное бюджетное образовательное учреждение }
				\vspace*{-0.7ex}
				\centerline{высшего образования}
				\vspace*{-0.7ex}
				\centerline{<<Московский государственный технический университет}
				\vspace*{-0.7ex}
				\centerline{имени Н.Э. Баумана}
				\vspace*{-0.7ex}
				\centerline{(национальный исследовательский университет)>>}
				\vspace*{-0.7ex}
				\centerline{(МГТУ им. Н.Э. Баумана)}}
		\end{minipage}
		%--------------------------------------
		
		%Полосы
		%--------------------------------------
		\vspace{-25pt}
		\hspace{-35pt}\rule{\textwidth}{2.3pt}
		
		\vspace*{-20.3pt}
		\hspace{-35pt}\rule{\textwidth}{0.4pt}
		%--------------------------------------
		
		\vspace{1.5ex}
		\hspace{-35pt} \noindent \small ФАКУЛЬТЕТ\hspace{80pt} <<Информатика и системы управления>>
		
		\vspace*{-16pt}
		\hspace{47pt}\rule{0.83\textwidth}{0.4pt}
		
		\vspace{0.5ex}
		\hspace{-35pt} \noindent \small КАФЕДРА\hspace{50pt} <<Теоретическая информатика и компьютерные технологии>>
		
		\vspace*{-16pt}
		\hspace{30pt}\rule{0.866\textwidth}{0.4pt}
		
		\vspace{11em}
		
		\begin{center}
			\Large {\bf Лабораторная работа № 5} \\
			\large {\bf по курсу <<Компьютерные сети>>} \\
			\large <<Импорт новостей в базу данных из RSS-канала>>
		\end{center}\normalsize
		
		\vspace{8em}
		
		
		\begin{flushright}
			{Студент группы ИУ9-32Б  Тараканов В. Д.. \hspace*{15pt}\\
				\vspace{2ex}
				Преподаватель Посевин Д. П.\hspace*{15pt}}
		\end{flushright}
		
		\bigskip
		
		\vfill
		
		
		\begin{center}
			\textsl{Москва 2024}
		\end{center}
	\end{titlepage}
	%--------------------------------------
	%		КОНЕЦ ТИТУЛЬНОГО ЛИСТА
	%--------------------------------------
	
	\renewcommand{\ttdefault}{pcr}
	
	\setlength{\tabcolsep}{3pt}
	\newpage
	\setcounter{page}{2}
	
	\section{Задание}\label{Sect::task}
	
Приложение должно подключаться к удаленной базе данных под управлением СУБД MySQL и выполнять обновление новостей в таблице. При этом важно учесть следующее: при повторном запуске приложения необходимо сравнивать новости в rss канале с теми, что записаны в таблице и при повторных запусках приложения дублей новостей в таблице быть не должно. Вывод обновления данных должен происходить асинхронно в html dashboard.
	\section{Результаты}\label{Sect::res}
	Исходный код программы представлен в листингах~\ref{lst:code1}--~\ref{lst:code2}.
	
	\begin{lstlisting}[language={},caption={client.go},label={lst:code1}]
		package main
		
	package main
	
	import (
	"database/sql"
	"fmt"
	"html/template"
	"log"
	"net/http"
	"time"
	
	_ "github.com/go-sql-driver/mysql"
	"github.com/gorilla/websocket"
	)
	
	type Article struct {
		ID      int
		Title   string
		Content string
	}
	
	var upgrader = websocket.Upgrader{
		CheckOrigin: func(r *http.Request) bool {
			return true
		},
	}
	
	func dbConn() (db *sql.DB, err error) {
		dsn := "iu9networkslabs:Je2dTYr6@tcp(students.yss.su:3306)/iu9networkslabs"
		db, err = sql.Open("mysql", dsn)
		if err != nil {
			return nil, err
		}
		err = db.Ping()
		if err != nil {
			return nil, err
		}
		return db, nil
	}
	
	func fetchArticles() ([]Article, error) {
		db, err := dbConn()
		if err != nil {
			return nil, err
		}
		defer db.Close()
		
		rows, err := db.Query("SELECT id, title, content FROM iu9Tarakanov")
		if err != nil {
			return nil, err
		}
		defer rows.Close()
		
		var articles []Article
		for rows.Next() {
			var article Article
			if err := rows.Scan(&article.ID, &article.Title, &article.Content); err != nil {
				return nil, err
			}
			articles = append(articles, article)
		}
		
		return articles, nil
	}
	
	func wsHandler(w http.ResponseWriter, r *http.Request) {
		conn, err := upgrader.Upgrade(w, r, nil)
		if err != nil {
			http.Error(w, "Failed to upgrade connection", http.StatusInternalServerError)
			return
		}
		defer conn.Close()
		
		for {
			articles, err := fetchArticles()
			if err != nil {
				log.Println("Error fetching articles:", err)
				return
			}
			
			err = conn.WriteJSON(articles)
			if err != nil {
				log.Println("Error writing to websocket:", err)
				return
			}
			
			time.Sleep(5 * time.Second)
		}
	}
	
	func main() {
		http.HandleFunc("/ws", wsHandler)
		http.HandleFunc("/", func(w http.ResponseWriter, r *http.Request) {
			tmpl, err := template.New("index").Parse(`
			<!DOCTYPE html>
			<html lang="en">
			<head>
			<meta charset="UTF-8">
			<meta name="viewport" content="width=device-width, initial-scale=1.0">
			<title>Articles</title>
			<script>
			let socket = new WebSocket("ws://" + window.location.host + "/ws");
			socket.onmessage = function(event) {
				let articles = JSON.parse(event.data);
				let container = document.getElementById("articles");
				container.innerHTML = "";
				articles.forEach(article => {
					let articleDiv = document.createElement("div");
					let title = document.createElement("h2");
					title.textContent = article.Title;
					let content = document.createElement("p");
					content.textContent = article.Content;
					articleDiv.appendChild(title);
					articleDiv.appendChild(content);
					articleDiv.appendChild(document.createElement("hr"));
					container.appendChild(articleDiv);
				});
			};
			</script>
			</head>
			<body>
			<h1>Articles</h1>
			<div id="articles"></div>
			</body>
			</html>
			`)
			if err != nil {
				http.Error(w, "Failed to create template", http.StatusInternalServerError)
				return
			}
			tmpl.Execute(w, nil)
		})
		
		fmt.Println("Server is running on 9786")
		log.Fatal(http.ListenAndServe(":9786", nil))
	}
	\end{lstlisting}
		\begin{lstlisting}[language={},caption={server.go},label={lst:code2}]
package main

import (
"database/sql"
"fmt"
"github.com/SlyMarbo/rss"
_ "github.com/go-sql-driver/mysql"
"log"
"strings"
"time"
)

var database *sql.DB

type news struct {
	title       string
	description string
}

func insertIntoDb(new news) {
	db, err := sql.Open("mysql", "iu9networkslabs:Je2dTYr6@tcp(students.yss.su:3306)/iu9networkslabs")
	if err != nil {
		log.Println(err)
	}
	database = db
	defer database.Close()
	query := database.QueryRow("SELECT EXISTS(SELECT `id` FROM `iu9Tarakanov` WHERE `title`=? OR `content`=?);", new.title, new.description)
	var isExists bool
	
	query.Scan(&isExists)
	
	if !isExists {
		database.Exec("INSERT INTO `iu9Tarakanov` (`title`, `content`) VALUES (?, ?);", new.title, new.description)
	} else {
		database.Exec("UPDATE `iu9Tarakanov` SET `content`=? WHERE `title`=?;", new.description, new.title)
		database.Exec("UPDATE `iu9Tarakanov` SET `title`=? WHERE `content`=?;", new.title, new.description)
	}
	
}

func rssparser() {
	rssObject, err := rss.Fetch("https://news.rambler.ru/rss/Namibia/")
	if err == nil {
		fmt.Printf("Title           : %s\n", rssObject.Title)
		fmt.Printf("Description     : %s\n", rssObject.Description)
		fmt.Printf("Link            : %s\n", rssObject.Link)
		fmt.Printf("Number of Items : %d\n", len(rssObject.Items))
		for v := range rssObject.Items {
			item := rssObject.Items[v]
			new_news := news{}
			new_news.title = strings.ReplaceAll(item.Title, "\u00A0", " ")
			new_news.description = strings.ReplaceAll(item.Summary, "\u00A0", " ")
			insertIntoDb(new_news)
			fmt.Println()
			fmt.Printf("Item Number : %d\n", v)
			fmt.Printf("Title       : %s\n", item.Title)
			fmt.Printf("Description : %s\n", item.Summary)
			
		}
	} else {
		fmt.Println(err)
	}
}

func main() {
	for {
		rssparser()
		time.Sleep(time.Second * 5)
	}
	
}
	\end{lstlisting}
	
	
	
	
	
\end{document}
